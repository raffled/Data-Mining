\documentclass[a4paper]{article}
\usepackage[top=.75in, bottom=.75in, left=.5in, right=.5in]{geometry}
\usepackage{amsmath}
\usepackage{amssymb}


\title{\vspace{-30pt}STAT 745 -- Fall 2014\\Assignment 12}
\author{Group 1: Francene Cicia, Doug Raffle, Melissa Smith}
\date{December 8, 2014}

\begin{document}
\setlength{\parindent}{0pt}
\vspace{-50pt}
\maketitle
\renewcommand{\thesection}{\alph{section}.}

\section{Graph Based Learning}
In graph based learning, we define and adjacency graph and response as:\\
$$
A = 
\begin{bmatrix}
     A_{LL} & A_{LU} \\
     A_{UL} & A_{UU} 
\end{bmatrix},
\quad
Y=
\begin{bmatrix}
     Y_{L} \\
     Y_{U}
\end{bmatrix}\\
$$\\

where $Y_{U}$ is missing. Let $S=D^{-1}A$, where $D$ is the diagonal
row sum matrix. So we have that the Laplcian matrix and $D$ can be
expressed as:\\ 

$$
D = 
\begin{bmatrix}
     D_{LL} + D_{LU} & 0 \\
     0 & D_{UU} + D_{UL}
\end{bmatrix},
\quad
\Delta=
\begin{bmatrix}
     \Delta_{LL} & \Delta_{LU} \\
     \Delta_{UL} & \Delta_{UU} 
\end{bmatrix}\\
=
\begin{bmatrix}
     D_{LL} + D_{LU} - A_{LL} & -A_{LU} \\
     -A_{UL} & D_{UU} + D_{UL} - A_{UU}
\end{bmatrix}\\
$$\\

From this, we employ propagation to fit $\hat{f}$. We have shown previously that
$\hat{f}_{L}= S_{LL}Y_{L} + S_{LU}(I - S_{UU})^{-1}S_{UL}Y_{L} =
M_{LL}Y_{L}$ and $\hat{f}_{U} = (I - S_{UU})^{-1}S_{UL}Y_{L}$. Let
$T_{LL}$ be the labeled only smoother. Also, define matrix $Q$ and
$S_{LUL}$ so that $M_{LL} = QT_{LL} + (I - Q)S_{LUL}$.  

\section{Exact form of $T_{LL}$}
Notice, we know that $S = D^{-1}A$. Using the matrices defined above,
we can solve for the partitions of $S$.  

$$
S= D^{-1}A = 
\begin{bmatrix}
     (D_{LL} + D_{LU})^{-1} & 0 \\
     0 & (D_{UU} + D_{UL})^{-1}
\end{bmatrix}\\
\quad
\begin{bmatrix}
     A_{LL} & A_{LU} \\
     A_{UL} & A_{UU} 
\end{bmatrix}\\
=
\begin{bmatrix}
     (D_{LL} + D_{LU})^{-1}A_{LL} & (D_{LL} + D_{LU})^{-1}A_{LU} \\
     (D_{UU} + D_{UL})^{-1}A_{UL} & (D_{UU} + D_{UL})^{-1}A_{UU} 
\end{bmatrix}\\
$$\\

We know that the supervised only smoother is $S_{LL}$, from the above matrix $S_{LL} = (D_{LL} + D_{LU})^{-1}A_{LL} = D_{LL}^{-1}A_{LL}$.\\
Thus, $T_{LL} = D_{LL}^{-1}A_{LL}$.\\

\section{Finding $\widetilde{f}_{U}$}
We have that $f_L = S_{LL}Y_L$.  To estimate the responses for the
unlabeled nodes, we need to use the information we have that connects
the labeled with the unlabeled cases.  We connect these cases with
$A_{UL}$ (and hence the smoother $S_{UL}$), so it follows that
$\widetilde{f}_U$ has the form: 

\begin{align*}
  \widetilde{f}_U = S_{UL}Y_L
\end{align*}


\section{Solving for $Q$ and $S_{LUL}$}

From $\hat{f}_{L}$, we can start with the fact that $M_{LL} = S_{LL} +
S_{LU}(I - S_{UU})^{-1}S_{UL}$. So, we can start by putting $M_{LL}$
in terms of partitions of $D$ and $A$:\\ 

\begin{align*}
M_{LL} =& S_{LL} + S_{LU}\left(I - S_{UU}\right)^{-1}S_{UL} \\
=&
  \left(D_{LL} + D_{LU}\right)^{-1}A_{LL}
  + \left(D_{LL} +
    D_{LU}\right)^{-1}A_{LU}\left(I - \left(D_{UU} +
      D_{UL}\right)^{-1}A_{UU}\right)^{-1}\left(D_{UU} +
    D_{UL}\right)^{-1}A_{UL}\\ 
=&
  \left(D_{LL}\left(I - D_{LL}^{-1}D_{LU}\right)\right)^{-1}A_{LL}
  +
  \left(D_{LL} + D_{LU}\right)^{-1}A_{LU}\left(I - \left(D_{UU} +
      D_{UL}\right)^{-1}A_{UU}\right)^{-1}\left(D_{UU} +
    D_{UL}\right)^{-1}A_{UL}\\ 
=&
  \left(I + D_{LL}^{-1}D_{LU}\right)^{-1}D_{LL}^{-1}A_{LL}
  +
  \left(D_{LL} + D_{LU}\right)^{-1}A_{LU}\left(\left(D_{UU} +
      D_{UL}\right)^{-1}\left(\left(D_{UU} + D_{UL}\right) -
      A_{UU}\right)\right)^{-1}\left(D_{UU} + D_{UL}\right)^{-1}A_{UL}\\ 
=&
  (I + D_{LL}^{-1}D_{LU})^{-1}D_{LL}^{-1}A_{LL}
  + \left(D_{LL} +
    D_{LU}\right)^{-1}A_{LU}\left(\left(D_{UU} +
      D_{UL}\right)^{-1}\Delta_{UU}\right)^{-1} \left(D_{UU} +
    D_{UL}\right)^{-1}A_{UL}\\ 
=&
  \left(I + D_{LL}^{-1}D_{LU}\right)^{-1}D_{LL}^{-1}A_{LL} +
  \left(D_{LL} + D_{LU}\right)^{-1}A_{LU}\Delta_{UU}^{-1}\left(D_{UU} +
    D_{UL}\right)\left(D_{UU} + D_{UL}\right)^{-1}A_{UL}\\ 
=&
  \left(I + D_{LL}^{-1}D_{LU}\right)^{-1}D_{LL}^{-1}A_{LL} +
  \left(D_{LL} + D_{LU}\right)^{-1}A_{LU}\Delta_{UU}^{-1}A_{UL}
\end{align*}\\

We know that:
\begin{equation*}
M_{LL} = S_{LL} + S_{LU}\left(I - S_{UU}\right)^{-1}S_{UL} = QT_{LL} + \left(I - Q\right)S_{LUL}
\end{equation*}\\
From the above derivation we found,
\begin{equation*}
M_{LL} = \left(I + D_{LL}^{-1}D_{LU}\right)^{-1}D_{LL}^{-1}A_{LL} +
\left(D_{LL} + D_{LU}\right)^{-1}A_{LU}\Delta_{UU}^{-1}A_{UL} 
\end{equation*}
It follows that,
\begin{equation*}
\left(I + D_{LL}^{-1}D_{LU}\right)^{-1}D_{LL}^{-1}A_{LL} +
\left(D_{LL} + D_{LU}\right)^{-1}A_{LU}\Delta_{UU}^{-1}A_{UL} =
QT_{LL} + \left(I - Q\right)S_{LUL} 
\end{equation*}\\
From this, we can see 
\begin{equation*}
\left(I + D_{LL}^{-1}D_{LU}\right)^{-1}D_{LL}^{-1}A_{LL} = QT_{LL}
\end{equation*}
From above, we know that$T_{LL} = D_{LL}^{-1}A_{LL}$ so it follows that:
\begin{align*}
Q = \left(I +D_{LL}^{-1}D_{LU}\right)^{-1}
\end{align*}

From the above equations for $M_{LL}$ we also have,
\begin{eqnarray}
\left(I - Q\right)S_{LUL} & = & \left(D_{LL} + D_{LU}\right)^{-1}A_{LU}\Delta_{UU}^{-1}A_{UL} \nonumber \\
& = & \left(D_{LU}\left(D_{LU}^{-1}D_{LL} + I\right)\right)^{-1}A_{LU}\Delta_{UU}^{-1}A_{UL} \nonumber \\
& = & \left(D_{LU}^{-1}D_{LL} + I \right)^{-1}D_{LU}^{-1}A_{LU}\Delta_{UU}^{-1}A_{UL} \nonumber 
\end{eqnarray}

In order for this to hold, we need that
$\left(I - Q\right) = \left(D_{LU}^{-1}D_{LL} + I \right)^{-1}$. 
Notice, $\left(I-Q\right)+Q = I$, we can see if this holds for the
$\left(I-Q\right)$ and $Q$ that we found.
\begin{eqnarray}
\left(I-Q\right) + Q &=& \left(D_{LU}^{-1}D_{LL} + I \right)^{-1} +
\left(I + D_{LL}^{-1}D_{LU}\right)^{-1} \nonumber \\ 
&=& \left(D_{LU}^{-1}\left(D_{LL} + D_{LU}\right)\right)^{-1} +
\left(D_{LL}^{-1}\left(D_{LL} + D_{LU}\right)\right)^{-1} \nonumber \\ 
&=& \left(D_{LL} + D_{LU}\right)^{-1} \left(D_{LU} + D_{LL}\right) \nonumber \\
&=& I
\end{eqnarray}
Therefore, we know that:
\begin{equation*}
S_{LUL} = D_{LU}^{-1}A_{LU}\Delta_{UU}^{-1}A_{UL} 
\end{equation*}\\
Also, by the partitioning of the graphs above,\\
\begin{eqnarray}
D_{LU}^{-1}A_{LU}\Delta_{UU}^{-1}A_{UL} &=& D_{LU}^{-1}\left(-\Delta_{LU}\right)\Delta_{UU}^{-1}\left(-\Delta_{UL}\right) \nonumber \\ 
&=& D_{LU}^{-1}\Delta_{LU}\Delta_{UU}^{-1}\Delta_{UL} \nonumber 
\end{eqnarray}\\

\section{Showing $S_{LUL}$ is right stochastic}

We will begin by showing some important identities. 
By properties of Laplacian matrices, we know $\Delta
\vec{1} = \vec{0}$ so,
\begin{equation*}
\begin{bmatrix}
     \Delta_{LL} & \Delta_{LU} \\
     \Delta_{UL} & \Delta_{UU} 
\end{bmatrix}
\vec{1}
\quad
=
\begin{bmatrix}
     D_{LL} + D_{LU} - A_{LL} & -A_{LU} \\
     -A_{UL} & D_{UU} + D_{UL} - A_{UU}
\end{bmatrix}\\
\vec{1}
\quad 
= 
\vec{0}
\end{equation*}

From above, we can see,
\begin{eqnarray}
\Delta_{UU}\vec{1} + \Delta_{UL}\vec{1} = \vec{0} \nonumber \\
\Delta_{UU}\vec{1} = -\Delta_{UL}\vec{1} \nonumber 
\end{eqnarray} \\
Similarly, we have,
\begin{eqnarray}
\Delta_{LL}\vec{1} + \Delta_{LU}\vec{1} &=& 0 \nonumber \\
\left(D_{LL} + D_{LU} - A_{LL}\right)\vec{1} + \Delta_{LU}\vec{1} &=& 0 \nonumber \\
D_{LL}\vec{1} - A_{LL}\vec{1} + D_{LU}\vec{1} + \Delta_{LU}\vec{1} &=& 0 \nonumber \\
\Delta_{LU}\vec{1} &=& -D_{LU}\vec{1} \nonumber 
\end{eqnarray}\\
Notice the above holds because $D_{LL}\vec{1} = A_{LL}\vec{1}$.\\

In order to show $S_{LUL}$ is right stochastic, we first need to
show $S_{LUL}\vec{1} = \vec{1}$.

\begin{eqnarray}
S_{LUL}\vec{1} & = & \left(D_{LU}^{-1}\Delta_{LU}\Delta_{UU}^{-1}\Delta_{UL}\right)\vec{1} \nonumber \\
& = & D_{LU}^{-1}\vec{1}\Delta_{LU}\vec{1}\Delta_{UU}^{-1}\vec{1}\Delta_{UL}\vec{1} \nonumber \\
& = &
D_{LU}^{-1}\vec{1}\left(-D_{LU}\right)\vec{1}\Delta_{UU}^{-1}\vec{1}\left(-\Delta_{UU}\right)\vec{1}
\nonumber \\ 
& = & \left(-I\right)\vec{1}\left(-I\right)\vec{1} \nonumber \\
& = & \vec{1} 
\end{eqnarray}\\
In order for $S_{LUL}$ to be right stochastic, $S_{LUL_{ij}} \geq
0$ for all $i$, $j$. We know that $A$ is an adjacency matrix, so all
of its elements, $A_{ij} \geq 0$. $D$ is the diagonal row sum matrix
of $A$, so $D_{ij} \geq A_{ij}$ for all $i$, $j$. This implies that
$\Delta_{ij} \geq 0$ for all $i$, $j$. $S_{LUL}$ is a product of
matrices containing all non-negative elements, thus $S_{LUL_{ij}} \geq
0$ for all $i$, $j$. Thus, $S_{LUL}$ is right stochastic.  


\section{Supervised vs. Semi-Supervised}
If we have no edges connecting the labeled to unlabed cases, $A_{UL}$
and hence $S_{UL}$ would provide no information about $Y_U$ and
$\widetilde{f}_U$ would not exist.  In this
case supervised and semi-supervised techniques should be equivalent,
since both cases would make the same predictions for the labeled cases and
no predictions for the unlabeled.\\

In the case that the data is linearly seperable, the boundary found by
both methods would similar and the results would be close.

\end{document}
