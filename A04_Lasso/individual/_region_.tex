\message{ !name(h4_ind.Rnw.tex)}
\message{ !name(h4_ind.Rnw) !offset(12) }

\setlength{\parindent}{0pt}
\vspace{-50pt}
\maketitle

\section{Variable Selection with The LASSO}
Generate Variables:\\
\begin{minipage}[t]{0.45\linewidth}
<<>>=
n <- 100
set.seed(10); x1 <- rnorm(n)
set.seed(20); x2 <- rnorm(n)
set.seed(30); x3 <- rnorm(n)
X <- cbind(1, x1, x2, x3)

colnames(X) <- c("int.",
                 paste(rep("x", 3),
                       1:3, sep=""))
beta <- as.matrix(c(2, 1, -1, -2))
set.seed(75); epsilon <- rnorm(n, 0, .9)
y <- X %*% beta + epsilon
@
\end{minipage}
\hfill
\vline
\hfill
\begin{minipage}[t]{0.45\linewidth}
<<>>=
get.z <- function(q){
    Z <- matrix(nrow=n, ncol=q)
    for(i in 1:q){
        set.seed(100*i)
        Z[,i] <- rnorm(n)
    }
    colnames(Z) <- paste(rep("z", q),
                         1:q, sep="")
}
@
\end{minipage}

\subsection{$q=0$}
\vspace{-20pt}
<<fig=FALSE, echo=FALSE>>=
library(lars, quietly=TRUE)
lasso.q0 <- lars(X, y, normalize=F)
@

\begin{minipage}[c]{0.5\linewidth}
<<fig=TRUE, echo=FALSE>>=
par(cex.lab=1.5, cex.axis=1.5, mar=c(5,5,5,2))
plot(lasso.q0, col=c("black", "darkred", "darkgreen"))
@
\end{minipage}
\hfill
\begin{minipage}[c]{0.5\linewidth}
<<fig=TRUE, echo=FALSE>>=
par(cex.lab=1.5, cex.axis=1.5, mar=c(5,5,5,.25))
cv.lars(X, y, normalize=F)
@
\end{minipage}

The LASSO is completed in three steps and keeps all three variables in the active set.
$\hat{\beta}_3$ is selected first, followed by $\hat{\beta}_1$ in the second step,
and finally $\hat{\beta}_2$.  As the number of iterations increases, the estimates
get closer to the known values.

The model has an optimal MSE when $||\beta||^1_1/\max ||\beta||^1_1=1$,
which is when we have all of our parameters.

\subsection{q=10}
\begin{minipage}[c]{0.45\linewidth}
<<>>=
lasso.q10 <- lars(cbind(X, get.z(10)), y, normalize=F)
plot(lasso.q10)
@
\end{minipage}
\hfill
\begin{minipage}[c]{0.5\linewidth}
<<fig=TRUE, echo=FALSE>>=
plot(lasso.q10, col=c("black", "darkred", "darkgreen"))
cv.lars(cbind(X, get.z(10)), y, 10, normalize=F)
@
\end{minipage}\\

\end{document}

\message{ !name(h4_ind.Rnw.tex) !offset(-89) }
